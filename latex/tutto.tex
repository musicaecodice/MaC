\documentclass[a4paper,11pt]{book}

%\usepackage[breakable]{tcolorbox}
\usepackage{parskip} 
   \providecommand{\tightlist}{\setlength{\itemsep}{0pt}
                               \setlength{\parskip}{0pt}}                              
\usepackage{geometry} % Used to adjust the document margins
    \geometry{verbose,tmargin=1in,bmargin=1in,lmargin=1in,rmargin=1in}
\usepackage{amsmath} % Equations
\usepackage{amssymb} % Equations
\usepackage{titling} % Toglie spazi esagerati ai titoli
\usepackage{graphicx}  % Per immagini
\usepackage[Export]{adjustbox} % Constrain images to a maximum size
    \adjustboxset{max size={0.9\linewidth}{0.9\paperheight}}
\usepackage{float}
    \floatplacement{figure}{H}
\usepackage{xcolor}
  \definecolor{mygreen}{rgb}{0,0.6,0}
  \definecolor{mygray}{rgb}{0.95,0.95,0.95}
  \definecolor{mymauve}{rgb}{0.58,0,0.82}  
  \definecolor{urlcolor}{rgb}{0,.145,.698}
  \definecolor{linkcolor}{rgb}{0,.145,.698} % .71,0.21,0.01
  \definecolor{citecolor}{rgb}{0,.145,.698} % .12,.54,.11
\usepackage{hyperref}
    \hypersetup{
      breaklinks=true,  % so long urls are correctly broken across lines
      colorlinks=true,
      urlcolor=urlcolor,
      linkcolor=urlcolor,
      citecolor=urlcolor,
      }  
      
\usepackage{listings} % per il codice

\lstset{ 
  backgroundcolor=\color{mygray},        
  breakatwhitespace=false,        
  breaklines=true,               
  captionpos=b,                   
  commentstyle=\color{mygreen},   
  escapeinside={\%*}{*)},        
  extendedchars=true,            
  firstnumber=1,               
  frame=single,	
  language=java,
  keepspaces=true,                
  basicstyle=\footnotesize\ttfamily,       
  numbers=left,                  
  numbersep=5pt,                 
  numberstyle=\tiny\color{black}, 
  rulecolor=\color{black},                        
  stepnumber=1,                        
  tabsize=2	                      
}

\setcounter{tocdepth}{3} % Include subsub section in indice
   
% ===============================================================
\graphicspath{
              {../1_premises/img/},
              {../2_instruments/img/},
              {../3_fixed/img/},
              {../4_cmusic/img/}
              {../5_cac/img/}
              {../6_lvset/img/}
              {../7_hyper/img/}
              {../8_lvcod/img/}
              }

\begin{document}

\author{Andrea Vigani}
\title{Music and code}
\maketitle

%\begin{abstract}
%In this Chapter bla bla bla.
%\end{abstract}

\chapter*{Preface}\label{prefazione}

This writing is not: 
\begin{itemize}
\tightlist
\item intended to be a book or a manual.
\item exhaustive.
\item written in a descriptive/discursive style.
\item aimed to provide absolute truths or certainties.
\end{itemize}

It aims to:
\begin{itemize}
\tightlist
\item provide readers with different aspects of electroacoustic music from different perspectives.
\item prompt readers to think individually or collectively about specific topics.
\item provide readers with input that will encourage them to delve deeper into the topics they find most interesting among those presented.
\item help readers build their own poetic consciousness and learn the most suitable informatic tools to achieve it.
\item help readers to find their own compositional models and procedures.
\end{itemize}

With the same purposes I use it as a framework in my University and Conservatory lessons.

\tableofcontents

\input{../1_premises/latex/1_premises.tex}
\input{../2_instruments/latex/2_instruments.tex}
\input{../3_fixed/latex/3_fixed.tex}
\input{../4_cmusic/latex/4_cmusic.tex}
\input{../5_cac/latex/5_cac.tex}
\input{../6_lvset/latex/6_lvset.tex}
\chapter{Mixed music and hyper-instruments}\label{mixed-music-and-hyper-instruments}

\section{Framework and historical context}\label{framework-and-historical-context}

Mixed music combines acoustic instruments and electronic elements.

Two different musical contexts:

\begin{enumerate}
\def\labelenumi{\arabic{enumi}.}
\tightlist
\item sound and formal augmentation (live composing).
\item acoustic instruments augmentation (hyper-instruments).
\end{enumerate}

For both of them we can do it in two ways:

\begin{itemize}
\tightlist
\item superimposing pre-composed electronic sounds and acoustic instruments.
\item acoustic instruments real-time sound processing.
\end{itemize}

For both we have usually:

\begin{itemize}
\tightlist
\item one electronic performer.
\item one sound director.
\item one or more instrumental performers.
\end{itemize}

\section{Sound and formal augmentation}\label{sound-and-formal-augmentation}

In this context we can:


\section{Software model}\label{software-model}
\subsection{Bus and Groups}\label{bus-and-groups}

\section{Feature extraction as control signals}\label{feature-extraction-as-control-signals}
\subsection{Envelope follower}\label{envelope-follower}
\subsection{Pitch follower}\label{pitch-follower}
\subsection{Centroid}\label{centroid}

\section{Classic live electronics sound processing}\label{classic-live-electronics-sound-processing}
\subsection{Ring modulation}\label{ring-modulation}
\subsection{Live recording and playback}\label{live-recording-and-playback}
\subsection{Granular synthesis}\label{granular-synthesis}
\subsection{Delay lines}\label{delay-lines}

\section{Composition sketches proposal}\label{composition-sketches-proposal}

\input{../8_lvcod/latex/8_lvcod.tex}

\lstlistoflistings
\end{document}